\documentclass[a4paper]{article}

\usepackage[spanish]{babel} % Le indicamos a LaTeX que vamos a escribir en espa�ol.
\usepackage[latin1]{inputenc} % Permite utilizar tildes y e�es normalmente
%\usepackage{framed}
\input{Algo1Macros}% Macros especificas para especificar problemas en AyEDI

\newcommand{\comen}[2]{%
\begin{framed}
\noindent \textsf{#1:} #2
\end{framed}
}
\begin{document}

\maketitle

\section{TAD Tipo es String}
\section{TAD Pokemones es Diccionario(Nat, Tipo)}
\section{TAD Coordenada es Tupla(Nat, Nat)}

\section{TAD Juego}
\begin{tipo}{Juego}
    \observador{mapa}{Juego}{(Mapa)}
    \observador{jugadores}{Juego}{conj(Jugador)}
    \observador{pokemones}{Juego}{Pokemones}
    \observador{posicionPokemon}{Nat n, Juego j}{Coordenada[n \in pokemones(j)]}
    \observador{posicionJugador}{Jugador j, Juego pGo}{Coordenada[j \in jugadores(pGo)]}
    \invariante[nuevoJuego]{Mapa \rightarrow Juego}
    \invariante[agJugador]{Jugador j, Coordenada c, Juego pGo \rightarrow Juego [esValida(c, pGo) \land L  puedeJugar(j,pGo)]}
    \invariante[agJugador]{Nat n, Tipo t, Coordenada c, Juego pGo \rightarrow Juego [n \notin pokemones(j) \land L esValidaPokemon(c, pGo)]}
    \invariante[informarPosicion]{Jugador j, Coordenada c, Juego pGo \rightarrow Juego [puedeMoverse(j, pGo) \land L esValidaEnJuego(c, pGo)]}
\end{tipo}

\end{document}
